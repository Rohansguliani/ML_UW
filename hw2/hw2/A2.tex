\documentclass{article}
\usepackage{amsmath} % For mathematical symbols
\usepackage{amssymb} % For additional symbols

\begin{document}

\section*{A2}

\subsection*{(I)}

The set is \textbf{not convex} because there exist points within the set whose connecting line segment partially lies outside the set. Specifically, selecting points \( b \) and \( c \), the straight line segment between them passes through the missing cutout region, which is not included in the set. Since a convex set must contain the entire line segment between any two of its points, this violation confirms that the set is non-convex.

\subsection*{(II)}

The set is \textbf{not convex} because there exist points within the set whose connecting line segment partially exits the set. Specifically, selecting points \( a \) and \( d \), the straight line segment between them passes through the indented cutout region, which is not part of the set. Since convexity requires that all such line segments remain entirely within the set, this counterexample proves that the set is non-convex.

\end{document}

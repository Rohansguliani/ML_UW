\documentclass{article}
\usepackage{amsmath} % For mathematical symbols and align environment
\usepackage{amssymb} % For additional symbols

\begin{document}

\section*{A8(d)}

Let \( \lambda \) be an eigenvalue of \( C \), and let \( v \) be the corresponding eigenvector. By definition of eigenvalues and eigenvectors, we have:
\[
C v = \lambda v.
\]
Taking the quadratic form \( v^\top C v \), we substitute \( C v = \lambda v \):
\[
v^\top C v = v^\top (\lambda v).
\]

Since \( \lambda \) is a scalar, we can factor it out:
\[
v^\top C v = \lambda (v^\top v).
\]
Since \( C \) is PSD, it satisfies:
\[
v^\top C v \geq 0 \quad \text{for all vectors } v.
\]

Substituting \( v^\top C v = \lambda (v^\top v) \), we get:
\[
\lambda (v^\top v) \geq 0.
\]
The term \( v^\top v \) is the squared norm of \( v \), which is strictly positive since \( v \neq 0 \) (by definition of an eigenvector). Thus:
\[
\lambda \geq 0.
\]
Thus, all eigenvalues of C are non-negative.


\hfill\(\blacksquare\)


\end{document}

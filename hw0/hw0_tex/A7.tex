\documentclass{article}
\usepackage{amsmath} % For mathematical symbols and align environment
\usepackage{amssymb} % For additional symbols

\begin{document}

\section*{A7}

\subsection*{(a)}
We expand \(f(x, y)\) as:
\[
f(x, y) = \sum_{i=1}^n \sum_{j=1}^n A_{ij} x_i x_j + \sum_{i=1}^n \sum_{j=1}^n B_{ij} y_i x_j + c.
\]

\subsection*{(b)}
The gradient of \(f(x, y)\) with respect to \(x\) is given by:
\[
\nabla_x f(x, y) = \begin{bmatrix}
\frac{\partial f}{\partial x_1} \\
\frac{\partial f}{\partial x_2} \\
\vdots \\
\frac{\partial f}{\partial x_n}
\end{bmatrix}
= 2A x + B^\top y.
\]

In summation form, the \(k\)-th component of the gradient is:
\[
\frac{\partial f}{\partial x_k} = 2 \sum_{j=1}^n A_{kj} x_j + \sum_{i=1}^n B_{ik} y_i.
\]

This is derived as follows:
\begin{align*}
\text{For the first term: } & \quad 
x^\top A x = \sum_{i=1}^n \sum_{j=1}^n A_{ij} x_i x_j. \\
& \text{When we take } \frac{\partial}{\partial x_k}, \text{ only terms where } i = k \text{ or } j = k \text{ contribute.} \\
& \text{If } i = k, \text{ the term becomes } A_{kj} x_k x_j, \text{ and } \frac{\partial}{\partial x_k}(A_{kj} x_k x_j) = A_{kj} x_j. \\
& \text{If } j = k, \text{ the term becomes } A_{ik} x_i x_k, \text{ and } \frac{\partial}{\partial x_k}(A_{ik} x_i x_k) = A_{ik} x_i. \\
& \text{Adding these contributions gives: } 2 \sum_{j=1}^n A_{kj} x_j. \\[10pt]
\text{For the second term: } & \quad 
y^\top B x = \sum_{i=1}^n \sum_{j=1}^n B_{ij} y_i x_j. \\
& \text{When we take } \frac{\partial}{\partial x_k}, \text{ only terms where } j = k \text{ contribute.} \\
& \text{If } j = k, \text{ the term becomes } B_{ik} y_i x_k, \text{ and } \frac{\partial}{\partial x_k}(B_{ik} y_i x_k) = B_{ik} y_i. \\
& \text{This gives: } \sum_{i=1}^n B_{ik} y_i. \\[10pt]
\text{For the third term: } & \quad 
c \text{ is a constant, so } \frac{\partial}{\partial x_k} c = 0. \\[10pt]
\text{Combining these results: } & \quad 
\frac{\partial f}{\partial x_k} = 2 \sum_{j=1}^n A_{kj} x_j + \sum_{i=1}^n B_{ik} y_i.
\end{align*}

\subsection*{(c)}
The gradient of \(f(x, y)\) with respect to \(y\) is given by:
\[
\nabla_y f(x, y) = \begin{bmatrix}
\frac{\partial f}{\partial y_1} \\
\frac{\partial f}{\partial y_2} \\
\vdots \\
\frac{\partial f}{\partial y_n}
\end{bmatrix}
= B x.
\]

In summation form, the \(k\)-th component of the gradient is:
\[
\frac{\partial f}{\partial y_k} = \sum_{j=1}^n B_{kj} x_j.
\]

This is derived as follows:
\begin{align*}
\text{For the first term: } & \quad 
x^\top A x = \sum_{i=1}^n \sum_{j=1}^n A_{ij} x_i x_j. \\
& \text{This term does not involve } y, \text{ so } \frac{\partial}{\partial y_k}(x^\top A x) = 0. \\[10pt]
\text{For the second term: } & \quad 
y^\top B x = \sum_{i=1}^n \sum_{j=1}^n B_{ij} y_i x_j. \\
& \text{When we take } \frac{\partial}{\partial y_k}, \text{ only terms where } i = k \text{ contribute.} \\
& \text{If } i = k, \text{ the term becomes } B_{kj} y_k x_j, \text{ and } \frac{\partial}{\partial y_k}(B_{kj} y_k x_j) = B_{kj} x_j. \\
& \text{This gives: } \sum_{j=1}^n B_{kj} x_j. \\[10pt]
\text{For the third term: } & \quad 
c \text{ is a constant, so } \frac{\partial}{\partial y_k} c = 0. \\[10pt]
\text{Combining these results: } & \quad 
\frac{\partial f}{\partial y_k} = \sum_{j=1}^n B_{kj} x_j.
\end{align*}

\end{document}

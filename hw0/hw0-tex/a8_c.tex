\documentclass{article}
\usepackage{amsmath} % For mathematical symbols and align environment
\usepackage{amssymb} % For additional symbols

\begin{document}

\section*{A8(c)}

We start with the following property of an invertible matrix:
\[
B B^{-1} = I,
\]
where \( I \) is the identity matrix.

Taking the transpose of both sides:
\[
(B B^{-1})^\top = I^\top.
\]

Using the property of transposes for matrix multiplication, \( (AB)^\top = B^\top A^\top \), this becomes:
\[
(B^{-1})^\top B^\top = I^\top.
\]

Since the transpose of the identity matrix is itself (\( I^\top = I \)), we have:
\[
(B^{-1})^\top B^\top = I.
\]

Because \( B \) is symmetric, \( B^\top = B \), so:
\[
(B^{-1})^\top B = I.
\]

By the definition of the inverse, this implies:
\[
(B^{-1})^\top = B^{-1}.
\]

Thus, \( B^{-1} \) is symmetric.

\hfill\(\blacksquare\)

\end{document}

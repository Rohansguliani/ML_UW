\documentclass{article}
\usepackage[margin=1in]{geometry}  % Helps prevent text from being cut off
\usepackage{amsmath}               % For mathematical symbols and align environment
\usepackage{amssymb}               % For additional symbols

\begin{document}

\section*{A7}

\subsection*{(a)}
We expand \(f(x, y)\) as:
\[
f(x, y) = \sum_{i=1}^n \sum_{j=1}^n A_{ij} x_i x_j 
        + \sum_{i=1}^n \sum_{j=1}^n B_{ij} y_i x_j
        + c.
\]

\subsection*{(b)}
The gradient of \(f(x, y)\) with respect to \(x\) is given by:
\[
\nabla_x f(x, y) = (A + A^\top)\,x \;+\; B^\top y.
\]
In summation form, the \(k\)-th component of the gradient is:
\[
\frac{\partial f}{\partial x_k} 
= \sum_{j=1}^n (A_{kj} + A_{jk})\,x_j 
  + \sum_{i=1}^n B_{ik}\,y_i.
\]
This arises because \(A\) is not necessarily symmetric; thus the partial derivatives from both \(A_{kj} x_k x_j\) and \(A_{ik} x_i x_k\) add up to \(\sum_j (A_{kj} + A_{jk}) x_j\), rather than \(2 \sum_j A_{kj} x_j\).

\paragraph{This is derived as follows:}
\begin{enumerate}
\item \textbf{For the term} \(\;x^\top A x = \sum_{i=1}^n \sum_{j=1}^n A_{ij} x_i x_j\):
  \[
  \text{When we take } \frac{\partial}{\partial x_k}, 
  \text{only terms where } i = k \text{ or } j = k \text{ contribute.}
  \]
  \[
  \text{If } i = k,\; A_{kj} x_k x_j 
    \quad\Longrightarrow\quad 
    \frac{\partial}{\partial x_k}(A_{kj} x_k x_j) = A_{kj} x_j.
  \]
  \[
  \text{If } j = k,\; A_{ik} x_i x_k 
    \quad\Longrightarrow\quad 
    \frac{\partial}{\partial x_k}(A_{ik} x_i x_k) = A_{ik} x_i.
  \]
  Adding these contributions:
  \[
    \sum_{j=1}^n A_{kj} x_j \;+\; \sum_{i=1}^n A_{ik} x_i 
    \;=\; \sum_{j=1}^n (A_{kj} + A_{jk})\,x_j,
  \]
  i.e.\ \(\bigl[(A + A^\top)\,x\bigr]_k\).

\item \textbf{For the term} \(\;y^\top B x = \sum_{i=1}^n \sum_{j=1}^n B_{ij} y_i x_j\):
  \[
  \text{When we take } \frac{\partial}{\partial x_k}, 
  \text{only terms where } j = k \text{ contribute.}
  \]
  \[
  \text{If } j = k,\; B_{ik} y_i x_k 
    \quad\Longrightarrow\quad 
    \frac{\partial}{\partial x_k}(B_{ik} y_i x_k) = B_{ik} y_i.
  \]
  Summing over \(i\) yields:
  \(\sum_{i=1}^n B_{ik} y_i\), i.e.\ \(\bigl[B^\top y\bigr]_k\).

\item \textbf{For the constant term} \(c\):
  \[
  \frac{\partial}{\partial x_k}\,c = 0.
  \]
\end{enumerate}
Combining these results for each component \(k\):
\[
\frac{\partial f}{\partial x_k} 
 = \bigl[(A + A^\top)\,x\bigr]_k \;+\; \bigl[B^\top y\bigr]_k.
\]
Hence in vector form:
\[
\nabla_x f(x, y) = (A + A^\top)\,x + B^\top y.
\]

\subsection*{(c)}
The gradient of \(f(x, y)\) with respect to \(y\) is:
\[
\nabla_y f(x, y) = B\,x.
\]
In summation form, the \(k\)-th component of the gradient is:
\[
\frac{\partial f}{\partial y_k} 
= \sum_{j=1}^n B_{kj} x_j.
\]
This follows because \(x^\top A x\) does not involve \(y\). Taking partials of \(y^\top B x\) with respect to \(y_k\) isolates exactly those terms \(B_{kj} x_j\).

\end{document}

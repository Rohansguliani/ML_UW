\documentclass{article}
\usepackage{amsmath} % For mathematical symbols and align environment
\usepackage{amssymb} % For additional symbols

\begin{document}

\section*{A8(a)}

We are given that:
\[
\text{diag}(v) 
\,=\, 
\begin{bmatrix}
v_1 & 0 & \cdots & 0 \\
0 & v_2 & \cdots & 0 \\
\vdots & \vdots & \ddots & \vdots \\
0 & 0 & \cdots & v_n
\end{bmatrix}
\quad \text{and} \quad
\text{diag}(w) 
\,=\, 
\begin{bmatrix}
w_1 & 0 & \cdots & 0 \\
0 & w_2 & \cdots & 0 \\
\vdots & \vdots & \ddots & \vdots \\
0 & 0 & \cdots & w_n
\end{bmatrix}.
\]

If we assume that \( \text{diag}(v)^{-1} = \text{diag}(w) \), it follows that:
\[
w_i = \frac{1}{v_i}, \quad \forall i \in \{1, 2, \ldots, n\}.
\]

We are also given that \( g(v_i) = w_i \), so substituting the relationship between \( v_i \) and \( w_i \), we find:
\[
g(v_i) = \frac{1}{v_i}.
\]
\hfill\(\blacksquare\)

\end{document}

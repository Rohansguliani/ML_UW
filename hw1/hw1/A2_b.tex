\documentclass{article}
\usepackage{amsmath} % For mathematical symbols and align environment
\usepackage{amssymb} % For additional symbols

\begin{document}

\section*{A2(b)}

To find the numerical estimate of \( \lambda \), we use the maximum likelihood estimate:
\[
\hat{\lambda} = \frac{1}{n} \sum_{i=1}^n x_i.
\]

The number of goals in the first five games is given as \( [2, 4, 6, 0, 1] \). Substituting these values:
\[
\hat{\lambda} = \frac{2 + 4 + 6 + 0 + 1}{5} = \frac{13}{5} = 2.6.
\]

Thus, the estimated \( \lambda \) is:
\[
\hat{\lambda} = 2.6.
\]

Next, we calculate the probability that the team scores 6 goals in their next game. The number of goals per game follows a Poisson distribution, with probability mass function:
\[
P(X = x \mid \lambda) = \frac{\lambda^x e^{-\lambda}}{x!}.
\]

Substituting \( x = 6 \) and \( \lambda = 2.6 \):
\[
P(X = 6 \mid \lambda = 2.6) = \frac{2.6^6 e^{-2.6}}{6!}.
\]

We compute this step by step:
1. Compute \( 2.6^6 \):
\[
2.6^6 = 308.915776.
\]

2. Compute \( e^{-2.6} \):
\[
e^{-2.6} \approx 0.07427.
\]

3. Compute \( 6! \):
\[
6! = 720.
\]

Substitute these values back:
\[
P(X = 6 \mid \lambda = 2.6) = \frac{308.915776 \cdot 0.07427}{720}.
\]

Simplify:
\[
P(X = 6 \mid \lambda = 2.6) \approx \frac{22.93885}{720} \approx 0.03187.
\]

Thus, the probability that the team scores 6 goals in their next game is approximately:
\[
P(X = 6 \mid \lambda = 2.6) \approx 0.03187 \quad \text{(3.187\%)}.
\]

\hfill\(\blacksquare\)

\end{document}
